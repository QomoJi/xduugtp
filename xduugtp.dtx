% \iffalse
%<*driver>
\ProvidesFile{xduugtp.dtx}
%</driver>
%<class>\NeedsTeXFormat{LaTeX2e}
%<class>\ProvidesClass{xduugtp}
%<*class>
[2022/03/09 v1.0.5 Xidian University Undergraduate Thesis Proposal class]
%</class>
%<*driver>
\documentclass{ctxdoc}
\usepackage{tocloft}
\setlength{\cftsubsecindent}{1.5em}
\setlength{\cftsubsubsecindent}{3em}
\setlength{\cftparaindent}{4.5em}
\setcounter{tocdepth}{4}
\setmonofont{Latin Modern Mono}
\usepackage{hologo}
\usepackage[os=win]{menukeys}
\renewmenumacro{\menu}[>]{roundedmenus}
\renewmenumacro{\directory}[/]{hyphenatepaths}
\usepackage{listings}
\lstset{
  backgroundcolor = \color{lightgray!30},
  keywordstyle    = \color{blue},
  stringstyle     = \color{brown},
  basicstyle      = {\small\ttfamily},
  breaklines      = true,
  tabsize         = 4,
  gobble          = 2,
  numbers         = left,
  numberstyle     = \tiny,
  frame           = single,
  xleftmargin     = \ccwd,
  numbersep       = \ccwd,
  columns         = fullflexible
}
\usepackage{fetamont}
\usepackage{xurl}
\usepackage{xspace}
\xspaceaddexceptions{。?!,、;:“”‘’—….--~·《》<>_}
\usepackage{multirow}
\usepackage{makecell}
% 交叉引用
\newcommand{\secref}[1]{第\xspace\ref{#1}\xspace 节}
\newcommand{\tableref}[1]{\tablename\xspace\ref{#1}\xspace}
% 文档类选项
\newcommand{\optx}[1]{\xspace\opt{#1}\xspace}
% /name LaTeX3控制序列
\newcommand{\csx}[1]{\xspace\cs{#1}\xspace}
% /name 传统LaTeX2e命令
\newcommand{\tnx}[1]{\xspace\tn{#1}\xspace}
%<name> LaTeX3 键值
\newcommand{\metax}[1]{\xspace\meta{#1}\xspace}
% {<name>} LaTeX2e 参数
\newcommand{\argx}[1]{\xspace\Arg{#1}\xspace}
% [<name>] LaTeX2e 可选参数
\newcommand{\oargx}[1]{\xspace\Arg{#1}\xspace}
% 文件
\newcommand{\filex}[1]{\xspace\texttt{#1}\xspace}
% 环境
\newcommand{\envx}[1]{\xspace\env{#1}\xspace}
% 宏包
\newcommand{\pkgx}[1]{\xspace\pkg{#1}\xspace}
% 文档类
\newcommand{\clsx}[1]{\xspace\cls{#1}\xspace}
% 目录
\newcommand{\dirx}[1]{\xspace\directory{#1}\xspace}
% 菜单
\newcommand{\menux}[1]{\xspace\menu{#1}\xspace}
% 值
\newcommand{\valuex}[1]{\xspace\texttt{#1}\xspace}
% 命令
\newcommand{\cmdx}[1]{\xspace\texttt{#1}\xspace}
% 链接
\newcommand{\footurl}[1]{\footnote{\url{#1}}}
\newcommand{\ctanurl}[1]{\href{https://mirrors.ustc.edu.cn/CTAN/#1}{\ttfamily CTAN://#1}}
\newcommand{\footctan}[1]{\footnote{\ctanurl{#1}}}
\newcommand{\htmlpre}[1]{\href{https://htmlpreview.github.io/?#1}{\ttfamily #1}}
% logo
\newcommand{\xduugtp}{{\bfseries\ffmfamily XDUUGTP}}
\newcommand{\texlive}{\TeX{} Live}
\newcommand{\mactex}{Mac\TeX{}}
\newcommand{\miktex}{\xspace\hologo{MiKTeX}\xspace}
\newcommand{\bibtex}{\hologo{BibTeX}}
% book
\newcommand{\lshortb}{《一份(不太)简短的\LaTeXe{}介绍》}
\newcommand{\insltxb}{《一份简短的关于\LaTeX{}安装的介绍》}
\newcommand{\symbolb}{《Comprehensive \LaTeX{} Symbol List》}
\begin{document}
\DocInput{\jobname.dtx}
\IndexLayout
\PrintChanges
\PrintIndex
\end{document}
%</driver>
% \fi
% \CheckSum{375}
% \CharacterTable
%  {Upper-case    \A\B\C\D\E\F\G\H\I\J\K\L\M\N\O\P\Q\R\S\T\U\V\W\X\Y\Z
%   Lower-case    \a\b\c\d\e\f\g\h\i\j\k\l\m\n\o\p\q\r\s\t\u\v\w\x\y\z
%   Digits        \0\1\2\3\4\5\6\7\8\9
%   Exclamation   \!     Double quote  \"     Hash (number) \#
%   Dollar        \$     Percent       \%     Ampersand     \&
%   Acute accent  \'     Left paren    \(     Right paren   \)
%   Asterisk      \*     Plus          \+     Comma         \,
%   Minus         \-     Point         \.     Solidus       \/
%   Colon         \:     Semicolon     \;     Less than     \<
%   Equals        \=     Greater than  \>     Question mark \?
%   Commercial at \@     Left bracket  \[     Backslash     \\
%   Right bracket \]     Circumflex    \^     Underscore    \_
%   Grave accent  \`     Left brace    \{     Vertical bar  \|
%   Right brace   \}     Tilde         \~}
% \DoNotIndex{\par,\\}
% \GetFileInfo{\jobname.dtx}
% \title{\bfseries\xduugtp{}文档类手册}
% \author{\href{https://github.com/note286/}{note286}}
% \date{\href{https://github.com/note286/xduugtp/releases/tag/\fileversion/}{\fileversion}~(\filedate)}
% \maketitle
% \thispagestyle{empty}
% \begin{abstract}
% \xduugtp{}文档类是面向西安电子科技大学本科生的
% 毕业论文(设计)开题报告\LaTeX{}模板,
% 支持\XeLaTeX{}编译方式,
% 支持Windows、macOS、GNU/Linux、Overleaf和TeXPage全平台,
% 支持插入签名图像,
% 支持参考文献引用,
% 具有丰富的说明文档和使用示例,
% 功能均有据可依,来源可靠。
% \end{abstract}
% \renewcommand\abstractname{免责声明}
% \begin{abstract}
% \noindent
% \begin{enumerate}
% \item 本模板的发布遵守
% \LaTeX{} Project Public License\footurl{https://www.latex-project.org/lppl.txt},
% 使用前请认真阅读协议内容。
% \item 本模板为作者根据西安电子科技大学教务处提供的
% Microsoft Word模板\footurl{https://jwc.xidian.edu.cn/info/1022/10853.htm}编写而成,
% 旨在供西安电子科技大学本科生撰写毕业论文(设计)开题报告使用。
% 建议用户使用前下载最新的Microsoft Word模板,
% 对比MD5\footnote{\ttfamily MD5:~58B9A670911EF51659F1CB86B45E03B8}是否有变化,
% 如有变化可以提Issue请求更新模板。
% \item 任何个人或组织以本模板为基础进行修改、扩展而生成的新的专用模板,
% 请严格遵守\LaTeX{} Project Public License,
% 由于违犯协议而引起的任何纠纷争端均与本模板作者无关。
% \end{enumerate}
% \end{abstract}
% \clearpage
% \tableofcontents
% \clearpage
% \section{模板介绍}
% \xduugtp{} (Xidian University Undergraduate Thesis Proposal)
% 是为了帮助西安电子科技大学本科生撰写毕业论文(设计)开题报告而编写的\LaTeX{}论文模板。
% \par
% 本文档将尽量完整的介绍模板的使用方法,
% 如有不清楚之处,或者想提出改进建议,
% 可以在GitHub Issues\footurl{https://github.com/note286/xduugtp/issues/}
% 参与讨论或提问。
% 注意,不同的问题开多个Issue,不要堆在一个Issue里。
% \par
% 虽然我感谢贡献,但我认为对于本模板,
% pull requests通常不是一个好方法(除非已经讨论并同意更改)。
% 模板的稳定性非常重要,这意味着对模板的更改必然非常保守。
% 这也意味着在进行任何更改之前必须进行大量讨论。
% 因此,如果您确实决定发布pull requests,请记住这一点:
% 我确实欣赏新想法,但不能总是将它们集成到模板中,
% 我很可能会拒绝以这种方式进行的更新。
% \par
% 模板的作用在于减少写作过程中格式调整的时间。
% 前提是遵守模板的用法,
% 否则即便用了\xduugtp{}也难以保证输出的文档符合学校规范。
% 此外,下面列出了一些常用的参考资料供用户学习:
% \begin{enumerate}
% \item \lshortb\footctan{info/lshort/chinese/lshort-zh-cn.pdf}
% \item \insltxb\footctan{info/install-latex-guide-zh-cn/install-latex-guide-zh-cn.pdf}
% \item \symbolb\footctan{info/symbols/comprehensive/symbols-a4.pdf}
% \end{enumerate}
% \par
% 用户如果遇到bug,或者发现与学校要求不一致,可以尝试以下办法:
% \begin{enumerate}
% \item 将模板升级到最新;
% \item 在GitHub Issues中报告bug。
% \end{enumerate}
% \section{贡献者}
% \xduugtp{}的开发过程中,主要的维护者为
% \href{https://github.com/note286/}{\ttfamily @note286}。
% 同时,也要感谢所有在GitHub和睿思\footurl{http://rs.xidian.edu.cn/}上提出问题的同学、老师们。
% \xduugtp{}的持续发展,离不开你们的帮助与支持。
% \section{致谢}
% 在学习文学编程的过程中,
% 《在\LaTeX{}中进行文学编程》\footurl{https://liam.page/2015/01/23/literate-programming-in-latex/}
% 和《Good things come in little packages: An introduction to writing .ins and .dtx files》\footurl{https://www.tug.org/TUGboat/tb29-2/tb92pakin.pdf}
% 提供了很大帮助,
% \filex{ctex.dtx}\footctan{language/chinese/ctex/ctex.dtx}提供了丰富的示例供学习参考。
% 在文档的编写过程中,参考了\filex{ctex.dtx}、
% \filex{install-latex-guide-zh-cn.tex}\footctan{info/install-latex-guide-zh-cn/install-latex-guide-zh-cn.tex}
% 和\filex{thuthesis.dtx}\footctan{macros/latex/contrib/thuthesis/thuthesis.dtx}。
% \section{安装与编译}
% 首先介绍\LaTeX{}套装安装,
% 其次介绍字体安装,
% 接着介绍模板下载与编辑,
% 最后介绍常见编译方式。
% \subsection{\LaTeX{}套装安装}
% \label{appins}
% 推荐Windows和GNU/Linux平台使用\texlive{},macOS平台使用\mactex{},跨版本升级均需要卸载旧版。
% \par
% 支持\miktex{},\miktex{}用户需要额外
% 安装Perl\footurl{https://strawberryperl.com/}以便于使用\cmdx{latexmk},
% 可通过运行\cmdx{perl -v}来检查是否已安装Perl。
% 此外,\miktex{}用户需自行打开\miktex{}控制台更新所有包。
% 本文档后续主要介绍\texlive{}与\mactex{}。
% \subsubsection{旧版本卸载}
% 本项目模板仅在最新版\texlive{}/\mactex{}通过测试,其他旧版本并未实际进行测试。
% 建议安装最新版\LaTeX{}发行版套装并更新所有包,
% 如果已安装\texlive{}或\mactex{}并且能够编译,用户可以选择不升级套装,不更新包。
% \par
% Windows平台卸载方法为管理员身份直接运行\dirx{C:/texlive/2021/tlpkg/installer/uninst.bat},
% 不同版本和安装位置请按需修改目录,更多介绍请参考\insltxb{}第1.2节,
% GNU/Linux平台卸载方法请参考\insltxb{}第2.2节,
% macOS上卸载方法请参考Uninstalling \mactex\footurl{https://www.tug.org/mactex/uninstalling.html}。
% \subsubsection{新版本安装}
% 校内睿思下载地址:
% \texlive{}\footurl{http://rs.xidian.edu.cn/forum.php?mod=viewthread&tid=1094234}
% 和\mactex{}\footurl{http://rs.xidian.edu.cn/forum.php?mod=viewthread&tid=1094235},
% 中科大源校外下载地址:
% \texlive\footctan{systems/texlive/Images/texlive.iso}
% 和\mactex\footctan{systems/mac/mactex/MacTeX.pkg}。
% \par
% 后续如无特殊情况,仅以Windows举例,其他操作系统上类似。
% 右键选择下载好的\filex{.iso}文件,选择\menux{打开方式>Windows资源管理器},
% 然后右键以管理员身份运行\filex{install-tl-windows.bat},保持默认配置即可。
% 如没有本地阅读文档的需求,安装时可以不勾选安装文档的选项,
% 这样会减少大约一半的磁盘占用空间,
% 具体来说,在\texlive{}安装窗口中点击左下角\menux{Advanced},
% 取消勾选\menux{安装字体/宏包文档目录树}和\menux{安装字体/宏包源码目录树}即可不安装文档和源码。
% 更多\LaTeX{}环境安装与配置请阅读\insltxb{}。
% \subsubsection{更新包管理器和所有包}
% 建议更新所有包至最新版,
% Windows平台上使用管理员身份运行cmd,
% GNU/Linux和macOS需要使用\cmdx{sudo}。
% 如果遇到更新失败,重新执行一遍。
% \begin{lstlisting}
% tlmgr repository set https://mirrors.ustc.edu.cn/CTAN/systems/texlive/tlnet/
% tlmgr update --all --self
% \end{lstlisting}
% \subsection{字体安装}
% \label{fontins}
% 考虑到可能存在版权问题,故不提供字体文件或字体下载链接。
% \par
% 可以通过运行以下命令来查看编译得到的pdf文件字体信息,
% 包括字体名称和字体嵌入等情况。
% \begin{lstlisting}
% pdffonts xduugtp.pdf
% \end{lstlisting}
% \subsubsection{Windows}
% Windows平台无需手动配置字体,所需字体Windows操作系统已内置。
% \subsubsection{GNU/Linux}
% \label{fontlinux}
% 由于默认情况下中易宋体的意大利形状对应的是中易楷体,
% 因此中文字体除中易宋体和中易黑体外,还需要中易楷体。
% \par
% 用户可以从Windows操作系统字体库中拷贝出\tableref{fontinfo}中所示的字体文件至GNU/Linux,
% 其中中文字体文件位于\dirx{C:/Windows/Fonts}处,
% Times New Roman字体文件位于\dirx{C:/Windows/Fonts/Times New Roman}处。
% 用户在查找字体时,可以根据\tableref{fontinfo}中所示的Windows系统内字体名称来查找,
% 找到后复制该字体,粘贴至某个空白文件夹即可得到对应的字体文件,
% 然后将所有的字体文件传输至GNU/Linux的\dirx{/usr/share/fonts}目录下。
% 然后就可以根据\secref{build}里的方法去编译了。
% \begin{table}
% \centering
% \caption{\xduugtp{}所需字体}
% \label{fontinfo}
% \begin{tabular}{crl}
% \toprule
% 字体名称 & 字体文件名 & Windows系统内字体名称 \\
% \midrule
% 中易黑体 & \filex{simhei.ttf} & SimHei Regular/黑体常规 \\
% 中易楷体 & \filex{simkai.ttf} & KaiTi Regular/楷体常规 \\
% 中易宋体 & \filex{simsun.ttc} & SimSun Regular/宋体常规 \\
% \multirowcell{4}{Times New Roman} & \filex{times.ttf} & Times New Roman Regular/常规 \\
% & \filex{timesbd.ttf} & Times New Roman Bold/粗体 \\
% & \filex{timesbi.ttf} & Times New Roman Bold Italic/粗斜体 \\
% & \filex{timesi.ttf} & Times New Roman Italic/斜体 \\
% \bottomrule
% \end{tabular}
% \end{table}
% \subsubsection{macOS}
% 参考\secref{fontlinux}从Windows平台提取字体文件,然后在macOS上双击安装字体文件即可。
% 注意,虽然macOS内置了Times New Roman字体,
% 但是该内置字体版本过于老旧,有缺字的现象,
% 建议将\tableref{fontinfo}中所示的字体文件全部安装。
% 然后就可以根据\secref{build}里的方法去编译了。
% \subsubsection{Overleaf}
% \label{overleaffont}
% 在Overleaf\footurl{https://cn.overleaf.com/}平台使用时,
% 由于Overleaf是安装在GNU/Linux上的最新版的\texlive{},
% 用户无需考虑\LaTeX{}套装版本问题,仅需要安装字体即可,
% 用户首先根据\secref{dledit}中的方法下载本仓库,
% 再根据\secref{fontlinux}中的方法得到字体文件。
% \par
% 在Overleaf左上角点击\menux{创建新项目},选择\menux{上传项目},
% 将压缩包上传至Overleaf,会自动进入该模板项目。
% 点击左上角\menux{新建目录}按钮,新建一个名为\dirx{fonts}的文件夹,
% 选中\dirx{fonts}文件夹,点击左上角\menux{上传}按钮将所有的字体文件上传。
% 最后根据\secref{overleafbuild}配置如何在线编译。
% \subsubsection{TeXPage}
% \label{texpagefont}
% 在TeXPage\footurl{https://www.texpage.com/}平台使用时,
% 由于TeXPage是安装在GNU/Linux上的最新版的\texlive{},
% 用户无需考虑\LaTeX{}套装版本问题,仅需要安装字体即可,
% 用户首先根据\secref{dledit}中的方法下载本仓库,
% 再根据\secref{fontlinux}中的方法得到字体文件。
% \par
% 在TeXPage首页左上角点击\menux{个人主页},进入个人主页后,点击左上角\menux{创建},
% 选择\menux{上传项目},将压缩包上传至TeXPage,进入该模板项目。
% 点击左上角\menux{新建文件夹}按钮,新建一个名为\dirx{fonts}的文件夹,
% 选中\dirx{fonts}文件夹,点击左上角\menux{上传文件}按钮将所有的字体文件上传。
% 最后根据\secref{texpagebuild}配置如何在线编译。
% \subsection{下载与编辑}
% \label{dledit}
% 请下载压缩包\footurl{https://github.com/note286/xduugtp/archive/refs/heads/main.zip},
% 当用户访问GitHub不便时,可以选择从国内GitHub镜像网站\footurl{https://gh.api.99988866.xyz/}下载压缩包。
% \par
% 用户可直接修改\filex{.tex}和\filex{.bib}等类型文件来进行相关内容的撰写。
% 此外,\filex{.cls}文件请不要修改。
% \par
% 其中,除指导教师意见和学院审核意见这两部分仅支持单页内容的撰写,其他部分均支持自动分页,无需用户手动分页。
% \subsection{编译}
% \label{build}
% 在参考\secref{appins}安装了\LaTeX{}套装,
% 且参考\secref{fontins}安装了缺失字体后,即可编译。
% 用户可以选择直接使用命令编译,
% 命令编译时切换到\filex{xduugtp.tex}所在目录执行命令即可。
% 或者选择合适的\LaTeX{} IDE来调用命令编译,
% IDE编译时选择\XeLaTeX{}编译方式,参考文献使用\bibtex{}编译。
% 关于PDF查看器,如果选择命令编译或\LaTeX{} IDE无内置PDF查看器,
% Windows平台上推荐使用Sumatra PDF Viewer\footurl{https://www.sumatrapdfreader.org/},
% macOS平台上推荐Skim\footurl{https://skim-app.sourceforge.io/},
% 适当配置即可支持正向同步和反向同步。
% \subsubsection{命令}
% 介绍如何使用命令编译,可选择使用latexmk来快速编译或者常规的四次编译。
% \par
% \cmdx{latexmk}编译:
% \begin{lstlisting}
% latexmk
% \end{lstlisting}
% \par
% 四次编译,即\XeLaTeX{}搭配\bibtex{}:
% \begin{lstlisting}
% xelatex -synctex=1 xduugtp
% bibtex xduugtp
% xelatex -synctex=1 xduugtp
% xelatex -synctex=1 xduugtp
% \end{lstlisting}
% \par
% 清理辅助文件:
% \begin{lstlisting}
% latexmk -c
% \end{lstlisting}
% \par
% \subsubsection{文本编辑器}
% 任何一款文本编辑器均可以编辑\filex{.tex}文件,
% 包括但不限于Sublime Text和Visual Studio Code等。
% 一些文本编辑器支持安装扩展,
% 例如Sublime Text可以安装LaTeXTools、
% Visual Studio Code可以安装LaTeX Workshop
% 来辅助进行\filex{.tex}文件的编辑,
% 还提供了一些常用的编译配置。
% 可以搭配Sumatra PDF Viewer或Skim实现反向同步,
% 正向同步一般需要文本编辑器或其扩展支持。
% \par
% 一些文本编辑器不支持自定义编译功能或者安装扩展,
% 依然可以使用文本编辑器来编辑\filex{.tex}文件,使用命令来进行编译。
% \subsubsection{WinEdt}
% 下载WinEdt\footurl{https://www.winedt.com/}安装包并安装,支持Windows平台。
% 安装后可以查看Quick Guide\footurl{http://www.winedt.com/download.html\#Quick\_Guide}获取更多关于WinEdt的使用帮助。
% \par
% 打开WinEdt后,点击\menux{File>Open}打开\filex{xduugtp.tex}文件。
% 点击\menux{Option>Execution Modes},在弹出的面板左侧选择\menux{TeXify},
% 在面板左下角点击\menux{Browse for executable},
% 依次找到\dirx{C:/texlive/2021/bin/win32/latexmk.exe}文件并点击打开,
% 如果安装\texlive{}至非默认目录,依情况修改;
% 将左下角的\menux{Switches}中对应值清空,最后点击面板上的\menux{OK}。
% \par
% 在\menux{Toolbar}中\menux{PDF Preview}左侧的按钮下拉菜单中可以切换编译引擎。
% 完全编译选择\menux{TeXify},可以自动处理交叉引用和参考文献引用,编译时间较长;
% 不考虑交叉引用和参考文献引用时,快速编译选择\menux{XeLaTeX},编译时间较短,
% 需要参考文献引用时再点击\menux{TeX>BibTeX}编译参考文献,
% 接着执行两次\menux{XeLaTeX}编译可以生成参考文献列表和参考文献引用。
% \par
% 点击\menux{Tools>Erase Output Files}或者\menux{Toolbar}中的\menux{Erase Output Files}按钮,
% 在弹出的面板中再点击\menux{Delete Now}可以清理辅助文件,
% 常用于某次报错后清理错误的辅助文件,避免二次报错。
% \par
% 可以参考QuickGuide.pdf\footurl{http://www.winedt.com/doc/QuickGuide.pdf}中第2.3节
% 配置WinEdt默认PDF查看器为Sumatra PDF Viewer,
% 即点击\menux{Option>Execution Modes},在弹出的面板选择\menux{PDF Viewer}标签,
% 将\menux{PDF Viewer Executable}改为\filex{SumatraPDF.exe},
% Sumatra PDF Viewer默认安装在\dirx{\%LOCALAPPDATA\%/SumatraPDF}处,
% 这样就可以使用Sumatra PDF Viewer来查看PDF文件。
% Sumatra PDF Viewer的反向同步一般WinEdt会自动配置,
% 如果需要手动配置,在Sumatra PDF Viewer左上角点击\menux{三道杠>设置>选项},
% 在最后设置反向搜索命令行中填写
% \begin{lstlisting}
% "C:\Program Files\WinEdt Team\WinEdt 10\WinEdt.exe" "[Open(|%f|);SelPar(%l,8);]"
% \end{lstlisting}
% 并点击确定即可。
% \par
% 注意,由于WinEdt添加新的编译配置较为复杂,
% 本方法将\menux{TeXify}内的编译引擎由\menux{LaTeX}改为\menux{latexmk},
% 并使用了主目录下的\filex{latexmkrc}编译配置。
% \subsubsection{TeXworks}
% 下载TeXworks\footurl{https://tug.org/texworks/}安装包并安装,支持Windows,GNU/Linux和macOS平台。安装后可以查看《A short manual for TeXworks》\footurl{https://github.com/TeXworks/manual/releases}获取更多关于TeXworks的使用帮助。
% \par
% 打开TeXworks后,点击\menux{编辑>首选项>排版>处理工具},
% 点击右下角蓝色加号,在弹出的面板中名称处填写\filex{latexmk},
% 程序处点击右侧\menux{浏览}选择\dirx{C:/texlive/2021/bin/win32/latexmk.exe}文件并点击打开,
% 如果安装\texlive{}至非默认目录,依情况修改,最后点击面板上的\menux{OK}。
% 选择新建的\menux{latexmk},点击右侧的蓝色上箭头移动至顶部,
% 再将内置的\menux{XeLaTeX}和\menux{BibTeX}移动至顶部,
% 使得\menux{latexmk}、\menux{XeLaTeX}和\menux{BibTeX}位于处理工具的顶部,方便后续切换引擎。
% 再选择下方的默认,可以将\menux{latexmk}或者\menux{XeLaTeX}设置为默认,最后点击\menux{OK}。
% \par
% 点击\menux{文件>打开},选择\filex{xduugtp.tex}文件,Toolbars左上角可以切换编译引擎。
% 完全编译选择\menux{latexmk},可以自动处理交叉引用和参考文献引用,编译时间较长;
% 不考虑交叉引用和参考文献引用时,快速编译选择\menux{XeLaTeX},编译时间较短,
% 需要参考文献引用时切换至\menux{BibTeX}编译参考文献,
% 接着执行两次\menux{XeLaTeX}编译可以生成参考文献列表和参考文献引用。
% \par
% 点击\menux{文件>删除辅助文件},在弹出的面板中再点击\menux{删除}可以清理辅助文件,
% 常用于某次报错后清理错误的辅助文件,避免二次报错。
% \par
% TeXworks内置了PDF查看器,支持正向同步和反向同步功能,
% 具体请查看《A short manual for TeXworks》中5.1节。
% \subsubsection{TeXstudio}
% 下载TeXstudio\footurl{https://www.texstudio.org/}安装包并安装,
% 支持Windows,GNU/Linux和macOS平台。
% 安装后可以查看《TeXstudio : User manual》\footnote{\htmlpre{https://github.com/texstudio-org/texstudio/master/utilities/manual/usermanual_en.html}}获取更多关于TeXstudio的使用帮助。
% \par
% 打开TeXstudio后,点击\menux{文件>打开},选择\filex{xduugtp.tex}文件。
% 点击\menux{选项>设置TeXstudio>命令},将\menux{Latexmk}处值改为\valuex{latexmk.exe},
% 切换至\menux{构建}标签,将默认编译器改为\menux{Latexmk}或者\menux{XeLaTeX}。
% \par
% TeXstudio无法快速切换编译引擎,只能在\menux{选项>设置TeXstudio>构建}里修改默认编译器,
% 或者点击\menux{工具>命令}里临时运行指定的编译引擎。
% 完全编译选择\menux{latexmk},可以自动处理交叉引用和参考文献引用,编译时间较长;
% 不考虑交叉引用和参考文献引用时,快速编译选择\menux{XeLaTeX},编译时间较短,
% 需要参考文献引用时切换至\menux{BibTeX}编译参考文献,
% 接着执行两次\menux{XeLaTeX}编译可以生成参考文献列表和参考文献引用。
% \par
% 点击\menux{工具>清理辅助文件},
% 在弹出的面板中选择合适的范围再点击\menux{OK}便可以清理辅助文件,
% 常用于某次报错后清理错误的辅助文件,避免二次报错。
% \par
% TeXstudio内置了PDF查看器,支持正向同步和反向同步功能,
% 具体请查看《TeXstudio : User manual》中4.10节。
% \subsubsection{Texmaker}
% 下载Texmaker\footurl{https://www.xm1math.net/texmaker/}安装包并安装,
% 支持Windows,GNU/Linux和macOS平台。
% 安装后可以查看《Texmaker : User manual》\footurl{https://www.xm1math.net/texmaker/doc.html}获取更多关于Texmaker的使用帮助。
% \par
% 打开Texmaker后,点击\menux{文件>打开},选择\filex{xduugtp.tex}文件。
% 点击\menux{选项>配置Texmaker>命令},
% 将\menux{LaTeX-Mk}中对应值改为\filex{latexmk},点击\menux{OK}。
% \par
% 工具栏中可以切换编译引擎。完全编译选择\menux{latexmk},
% 可以自动处理交叉引用和参考文献引用,编译时间较长;
% 不考虑交叉引用和参考文献引用时,快速编译选择\menux{XeLaTeX},编译时间较短,
% 需要参考文献引用时切换至\menux{BibTeX}编译参考文献,
% 接着执行两次\menux{XeLaTeX}编译可以生成参考文献列表和参考文献引用。
% \par
% 点击\menux{工具>清除历史记录},在弹出的面板中再点击\menux{删除文件}可以清理辅助文件,
% 常用于某次报错后清理错误的辅助文件,避免二次报错。
% \par
% Texmaker内置了PDF查看器,支持正向同步和反向同步功能,
% 具体请查看《Texmaker : User manual》中3.3节。
% \subsubsection{Overleaf}
% \label{overleafbuild}
% 用户首先根据\secref{overleaffont}中关于字体安装的介绍安装好字体,
% 再点击左上角的\menux{菜单}按钮修改编译器为\menux{XeLaTeX},
% 最后为\clsx{xduugtp}文档类传入\optx{overleaf}参数,
% 即将\filex{xduugtp.tex}中
% \begin{lstlisting}
% \documentclass{xduugtp}
% \end{lstlisting}
% 改为
% \begin{lstlisting}
% \documentclass[overleaf]{xduugtp}
% \end{lstlisting}
% 后即可正常编译。
% \subsubsection{TeXPage}
% \label{texpagebuild}
% 用户首先根据\secref{texpagefont}中关于字体安装的介绍安装好字体,
% 再点击右上角的\menux{设置}按钮修改\menux{LaTeX编译器}为\menux{XeLaTeX},
% 最后为\clsx{xduugtp}文档类传入\optx{texpage}参数,即将\filex{xduugtp.tex}中
% \begin{lstlisting}
% \documentclass{xduugtp}
% \end{lstlisting}
% 改为
% \begin{lstlisting}
% \documentclass[texpage]{xduugtp}
% \end{lstlisting}
% 后即可正常编译。
% \section{使用说明}
% 本手册假定用户已经能处理一般的\LaTeX{}文档,并对\BibTeX{}有一定了解。
% 如果从未接触过\LaTeX{},建议先学习相关的基础知识。
% 想要入门\LaTeX{}或者对\LaTeX{}语法一知半解的请阅读\lshortb{},
% 想要查询数学符号的可以在\symbolb{}中搜索。
% 本项目文档都很详细,请认真阅读本文档。
% 由于模板升级频繁,在开始使用和提问前,
% 请确保您已经认真完整地阅读了本文档和示例代码。
% \subsection{模板选项}
% \begin{function}[added=2022-01-02]{overleaf}
% 开启Overleaf平台支持,详情请参考\secref{overleafbuild}。
% \end{function}
% \begin{function}[added=2022-01-02]{texpage}
% 开启TeXPage平台支持,详情请参考\secref{texpagebuild}。
% \end{function}
% \subsection{参考文献引用}
% \label{refcite}
% 在btxdoc\footctan{biblio/bibtex/base/btxdoc.pdf}文档第3.1章节
% 描述了\bibtex{}的标准类型的必选域和可选域。
% 在\secref{refcite}中提到本模板采用的是GB/T 7714-2015,用户可以自行下载相应标准查看示例。
% 国标中规定了16种参考文献类型,部分类型不是\bibtex{}的标准类型,
% 此外国标中规定的著录项目多于\bibtex{}的标准域。
% 因此,强烈建议用户参考gbt7714.pdf\footctan{biblio/bibtex/contrib/gbt7714/gbt7714.pdf}中
% 关于文献类型和著录项目的描述以及standard.bib\footurl{https://github.com/zepinglee/gbt7714-bibtex-style/blob/master/test/testbst/support/standard.bib}中丰富的示例。
% \par
% 百度学术\footurl{https://xueshu.baidu.com/}
% 和Google Scholar\footurl{https://scholar.google.com.hk/}
% 导出的bib文件不是很规范,经常有很大问题,
% 感兴趣的可以去\bibtex{} format explained\footurl{https://www.bibtex.com/g/bibtex-format/}了解bib文件的合法格式。
% 用户可以使用dblp\footurl{https://dblp.org/}生成的bib条目,
% 遇到dblp没有的条目,可以参考上述文档和示例自行整理。
% 已添加部分常用类型参考文献条目样例至\filex{xduugtp.bib}供用户参考。
% 需要注意的是,不要轻易使用分组即\valuex{\{\}},尤其是\valuex{author}域。
% 无论中英文,每个作者均使用\valuex{and}连接。
% 除非文献卷号、期号和页码均无,否则不必提供DOI域。
% 对于网页链接,使用\valuex{online}类型,
% 填写\valuex{author}、\valuex{title}、\valuex{url}和\valuex{urldate}域即可。
% \par
% 在开题报告表中,一般仅国内外研究现状处会出现参考文献引用,
% 因此用户在撰写国内外研究现状时可以直接引用参考文献,
% 对应的参考文献列表会自动出现在国内外研究现状后,无需用户干预。
% 本模板已根据学校要求设置了\tnx{cite}生成的引用样式,
% 直接使用即可符合学校的要求,例如:
% \begin{lstlisting}
% 测试引用\cite{ChangHTD19,WangZSS21,GongL21}是否正常。
% \end{lstlisting}
% \subsection{字体形状与字体系列}
% 本项目模板正文默认使用中易宋体和Times New Roman,
% 支持常用的字体形状如意大利和倾斜,
% 支持常见的字体系列如加宽加粗。
% \par
% 对于中易宋体,意大利形状对应中易楷体,
% 倾斜形状对应中易宋体伪斜体,
% 加宽加粗系列对应中易宋体伪粗体。
% 其中,参考fontspec.pdf\footctan{macros/unicodetex/latex/fontspec/fontspec.pdf}中的示例,
% 设置倾斜程度为$0.2$,
% 参考清华大学学位论文模板thuthesis.dtx\footctan{macros/latex/contrib/thuthesis/thuthesis.dtx},
% 设置粗细程度为$3$。
% \par
% 对于Times New Roman,意大利形状及加宽加粗系列均有对应的Times New Roman字体文件,
% 倾斜形状与意大利形状一致,因此无需伪斜体和伪粗体。
% \par
% 字体形状和字体系列可以组合使用,例如:
% \begin{lstlisting}
% 意大利形状\textit{测试ABCabc123}
% 倾斜形状\textsl{测试ABCabc123}
% 加宽加粗系列\textbf{测试ABCabc123}
% 加宽加粗系列叠加意大利形状\textbf{\textit{测试ABCabc123}}
% 加宽加粗系列叠加倾斜形状\textbf{\textsl{测试ABCabc123}}
% 强调\emph{测试ABCabc123}
% \end{lstlisting}
% \subsection{交叉引用}
% 本项目模板有图、表、和公式等引用命令,使用方法如下:
% \begin{lstlisting}
% 图的具体内容如\figureref{figu1}所示。
% 表的具体内容如\tableref{tabl1}所示。
% 公式的具体内容如\equationref{equa1}所示。
% \end{lstlisting}
% \subsection{图片}
% 图片插入时,如果将图片文件放入\dirx{figures}文件夹,
% 则无需添加路径,直接使用图片文件名即可,甚至扩展名也可以省略不写,可以参考如下示例:
% \begin{lstlisting}
% \begin{tpfigure}
% \includegraphics[width=.3\linewidth]{fig1file}
% \captionof{figure}{方案开销}
% \label{fig1}
% \end{tpfigure}
% \end{lstlisting}
% 如果用户需要插入多页pdf文件的某一页,可以使用page参数,例如插入\filex{figfile.pdf}的第2页:
% \begin{lstlisting}
% \includegraphics[page=2]{figfile}
% \end{lstlisting}
% 另外,本项目模板实测\tnx{textwidth}为\valuex{417.11752pt},\tnx{textheight}为\valuex{700.50723pt},
% 对插图字号有要求的用户画图时可参考这两个数值,避免图片尺寸超过页面可编辑范围。
% \par
% 此外,由于开题报告的特殊性,不支持浮动体,
% 本项目模板自定义\envx{tpfigure}环境来插入图片,图片显示位置即插入位置。
% \par
% 对于图片的格式,优先推荐\filex{.tikz}、\filex{.pgf}和\filex{.pdf}格式的图片,不推荐\filex{.png}和\filex{.jpg}等非矢量图片格式。此外,对于已有的\filex{.pdf}格式的图片,不需要转换成\filex{.eps}文件。针对Microsoft Visio等绘图软件,建议使用打印成\filex{.pdf}的方式,再使用\texlive{}自带的\cmdx{pdfcrop}命令进行快速高效裁剪。其中,使用\filex{.tikz}和\filex{.pgf}格式的图片时,用户需要使用\tnx{input}命令而不是\tnx{includegraphics}命令。
% \subsection{表格}
% 参考\filex{西安电子科技大学研究生学位论文模板(2015年修订版)-2019.03修订.docx}中关于表格字号的要求,本项目模板已经重定义了表格字号大小为5号,用户无需手动指定字号,可以参考如下示例:
% \begin{lstlisting}
% \begin{tptable}
% \renewcommand{\arraystretch}{1.5}
% \captionof{table}{这是一个表格}
% \label{tabl1}
% \begin{tabular}{|c|c|}
% \hline
% 表格 & 表格 \\
% \hline
% 表格 & 表格 \\
% \hline
% \end{tabular}
% \end{tptable}
% \end{lstlisting}
% 由于开题报告的特殊性,不支持浮动体,本项目模板自定义\envx{tptable}环境来插入表格,表格显示位置即插入位置。
% \subsection{签名图像}
% 由于部分用户线上进行开题,教师无法现场手写签名,故支持签名图像替代手写签名。
% \par
% 用户需要自行制作好签名图像,推荐处理成字迹全黑且背景透明并以\filex{.png}格式存储,使用纯白色背景并以其他格式如\filex{.jpg}和\filex{.pdf}等格式存储也可。此外需要将图片四周的空白裁掉,尽量减小字迹与四周的间距。将准备好的签名图像放入\dirx{figures/sign}目录下。
% \par
% 用户在\filex{xduugtp.tex}中将签名图像相关的以\tnx{renewcommand}开头的语句取消注释,现场手写签名的保持注释状态即可。每个签名图像使用\tnx{sign}添加,其中分组内为签名图像文件名,无需扩展名且无需路径。
% \par
% 此外,如果需要打印好再线下填写日期,保持日期相关语句注释状态不变即可。如果需要插入日期手写图像,将对应的日期插入语句取消注释即可,如
% \begin{lstlisting}
% \renewcommand\zdjsqmrq{\sign{lisidate}}
% \end{lstlisting}
% 如果需要插入电子版的日期,将对应的日期插入语句取消注释即可,如
% \begin{lstlisting}
% \renewcommand\zdjsqmrq{2022年1月10日}
% \end{lstlisting}
% \StopEventually{}
% \section{代码实现}
% \changes{v1.0.0}{2022/02/01}{使用dtx分发}
% \setlength\parindent{0pt}
% \subsection{xduugtp.cls}
%    \begin{macrocode}
%<*class>
%    \end{macrocode}
% \changes{v0.1.0}{2022/01/02}{新增西电本科生开题报告\XeLaTeX{}模板}
% \subsubsection{定义选项}
% 设置运行平台为Overleaf。
%    \begin{macrocode}
\newif\ifoverleaf\overleaffalse
\DeclareOption{overleaf}{\overleaftrue}
%    \end{macrocode}
% 设置运行平台为TeXPage。
%    \begin{macrocode}
\DeclareOption{texpage}{\overleaftrue}
%    \end{macrocode}
%    \begin{macrocode}
\ProcessOptions\relax
%    \end{macrocode}
% 加载\clsx{ctexart}文档类,设置纸张大小为A4,设置默认字号为小四。
%    \begin{macrocode}
\LoadClass[a4paper,zihao=-4,fontset=none]{ctexart}
%    \end{macrocode}
% \subsubsection{装载宏包}
% 加载\pkgx{hyperref}宏包,用于书签增加章节序号,并隐藏超链接颜色和边框。
%    \begin{macrocode}
\RequirePackage[bookmarksnumbered,hidelinks]{hyperref}
%    \end{macrocode}
% \changes{v0.1.2}{2022/01/03}{支持URL自动断行}
% 加载\pkgx{xurl}宏包,用于URL自动断行。
%    \begin{macrocode}
\RequirePackage{xurl}
%    \end{macrocode}
% 加载\pkgx{tcolorbox}宏包,用于绘制边框。
%    \begin{macrocode}
\RequirePackage[most]{tcolorbox}
%    \end{macrocode}
% 加载\pkgx{xifthen}宏包,用于条件判断。
%    \begin{macrocode}
\RequirePackage{xifthen}
%    \end{macrocode}
% 加载\pkgx{xeCJKfntef}宏包,用于设置下划线。
%    \begin{macrocode}
\RequirePackage{xeCJKfntef}
%    \end{macrocode}
% 加载\pkgx{geometry}宏包,用于设置页边距。
%    \begin{macrocode}
\RequirePackage[left=3.17cm,right=3.17cm,top=2.54cm,bottom=2.54cm,footskip=0cm,headsep=0cm,headheight=0cm]{geometry}
%    \end{macrocode}
% 加载\pkgx{natbib}宏包和\pkgx{gbt7714}宏包,用于支持参考文献引用。
% \changes{v0.1.4}{2022/01/03}{适配旧版\pkgx{gbt7714}}
%    \begin{macrocode}
\RequirePackage[sort&compress,square,super,comma]{natbib}
\RequirePackage{gbt7714}
\@ifpackagelater{gbt7714}{2020/03/04}
  {\PassOptionsToPackage{numbers}{natbib}}
  {\PassOptionsToPackage{numbers}{gbt7714}}
\newcommand\ckwx{}
\@ifpackagelater{gbt7714}{2020/03/04}
  {\renewcommand\ckwx{\bibliographystyle{gbt7714-numerical}}}
  {}
\setlength{\bibsep}{0pt}
%    \end{macrocode}
% 加载\pkgx{graphicx}宏包,用于支持图片插入,并配置默认图片目录。
% \changes{v0.3.0}{2022/01/10}{增加手写日期图像目录}
%    \begin{macrocode}
\RequirePackage{graphicx}
\graphicspath{{figures/}{figures/sign/}}
%    \end{macrocode}
% \changes{v0.2.0}{2022/01/09}{支持插入图表}
% 加载\pkgx{caption}宏包,用于设置图表标题。
%    \begin{macrocode}
\RequirePackage[labelsep=quad]{caption}
\setlength{\abovecaptionskip}{6bp}
\setlength{\belowcaptionskip}{6bp}
\renewcommand{\captionfont}{\zihao{5}}
%    \end{macrocode}
% 加载\pkgx{xspace}宏包,用于自动添加空格。
%    \begin{macrocode}
\RequirePackage{xspace}
\xspaceaddexceptions{。?!,、;:“”‘’—….--~·《》<>_}
%    \end{macrocode}
% \begin{macro}{\figurename,\tablename,\figureref,\tableref,\equationref}
% 自定义交叉引用命令。
%    \begin{macrocode}
\renewcommand{\figurename}{图}
\renewcommand{\tablename}{表}
\newcommand{\figureref}[1]{图\xspace\ref{#1}\xspace}
\newcommand{\tableref}[1]{表\xspace\ref{#1}\xspace}
\newcommand{\equationref}[1]{公式(\ref{#1})}
%    \end{macrocode}
% \end{macro}
% \begin{environment}{tpfigure}
% 定义图片环境。
%    \begin{macrocode}
\NewEnviron{tpfigure}{
  \begin{minipage}{\textwidth}
  \vspace*{6bp}
  \centering
  \BODY
  \end{minipage}
}
%    \end{macrocode}
% \end{environment}
% \begin{environment}{tptable}
% 定义表格环境。
%    \begin{macrocode}
\NewEnviron{tptable}{
  \begin{minipage}{\textwidth}
  \centering
  \zihao{5}
  \BODY
  \vspace*{6bp}
  \end{minipage}
}
%    \end{macrocode}
% \end{environment}
% 设置PDF元数据。
%    \begin{macrocode}
\BeforeBeginEnvironment{document}{
  \hypersetup{
    pdfauthor={\authorName},
    pdfsubject={西安电子科技大学\school{}学院\class{}届本科生毕业论文(设计)开题报告}
  }
}
%    \end{macrocode}
% \begin{macro}{\sign}
% 插入签名图片。
%    \begin{macrocode}
\newcommand\sign[1]{\includegraphics[height=1.5em,keepaspectratio]{#1}}
%    \end{macrocode}
% \end{macro}
% \begin{macro}{\zdjsqm}
% 指导教师签名。
%    \begin{macrocode}
\newcommand\zdjsqm{}
%    \end{macrocode}
% \end{macro}
% \begin{macro}{\xyshqm}
% 学院审核签名。
%    \begin{macrocode}
\newcommand\xyshqm{}
%    \end{macrocode}
% \end{macro}
% \changes{v0.3.0}{2022/01/10}{支持插入手写日期图像}
% \begin{macro}{\zdjsqmrq}
% 指导教师签名日期。
%    \begin{macrocode}
\newcommand\zdjsqmrq{年\qquad 月\qquad 日}
%    \end{macrocode}
% \end{macro}
% \begin{macro}{\xyshqmrq}
% 学院审核签名日期。
%    \begin{macrocode}
\newcommand\xyshqmrq{年\qquad 月\qquad 日}
%    \end{macrocode}
% \end{macro}
% 设置页面样式为空。
%    \begin{macrocode}
\pagestyle{empty}
%    \end{macrocode}
% \subsubsection{配置字体}
% \begin{macro}{\FakeBoldValue}
% 参考清华大学学位论文模板thuthesis.dtx\footctan{macros/latex/contrib/thuthesis/thuthesis.dtx},设置粗细程度为$3$。
%    \begin{macrocode}
\newcommand\FakeBoldValue{3}
%    \end{macrocode}
% \end{macro}
% \begin{macro}{\FakeSlantValue}
% 参考fontspec.pdf\footctan{macros/unicodetex/latex/fontspec/fontspec.pdf}中的示例,设置倾斜程度为$0.2$。
%    \begin{macrocode}
\newcommand\FakeSlantValue{0.2}
%    \end{macrocode}
% \end{macro}
% \paragraph{非安装字体}
%    \begin{macrocode}
\ifoverleaf
%    \end{macrocode}
% 默认英文字体为Times New Roman。
%    \begin{macrocode}
  \setmainfont{times.ttf}
    [Path=fonts/,
    BoldFont=timesbd.ttf,
    ItalicFont=timesi.ttf,
    BoldItalicFont=timesbi.ttf]
%    \end{macrocode}
% 默认中文字体为中易宋体。
%    \begin{macrocode}
  \setCJKmainfont{simsun.ttc}
    [Path=fonts/,
    BoldFont={simsun.ttc},BoldFeatures={FakeBold=\FakeBoldValue},
    SlantedFont={simsun.ttc},SlantedFeatures={FakeSlant=\FakeSlantValue},
    BoldSlantedFont={simsun.ttc},BoldSlantedFeatures={FakeBold=\FakeBoldValue,FakeSlant=\FakeSlantValue},
    ItalicFont={simkai.ttf},
    BoldItalicFont={simkai.ttf},BoldItalicFeatures={FakeBold=\FakeBoldValue}]
%    \end{macrocode}
% 黑体字体。
% \changes{v0.1.5}{2022/01/03}{修正Overleaf中标题年份数字字体}
% \changes{v0.1.5}{2022/01/03}{定义英文黑体字体族}
%    \begin{macrocode}
  \newCJKfontfamily\heiti{simhei.ttf}[Path=fonts/,BoldFont={simhei.ttf},BoldFeatures={FakeBold=\FakeBoldValue}]
  \newfontfamily\enheiti{simhei.ttf}[Path=fonts/,BoldFont={simhei.ttf},BoldFeatures={FakeBold=\FakeBoldValue}]
%    \end{macrocode}
% \paragraph{已安装字体}
%    \begin{macrocode}
\else
%    \end{macrocode}
% 默认英文字体为Times New Roman。
%    \begin{macrocode}
  \setmainfont{Times New Roman}
%    \end{macrocode}
% 默认中文字体为中易宋体。
%    \begin{macrocode}
  \setCJKmainfont{SimSun}[
    BoldFont={SimSun},BoldFeatures={FakeBold=\FakeBoldValue},
    SlantedFont={SimSun},SlantedFeatures={FakeSlant=\FakeSlantValue},
    BoldSlantedFont={SimSun},BoldSlantedFeatures={FakeBold=\FakeBoldValue,FakeSlant=\FakeSlantValue},
    ItalicFont={KaiTi},
    BoldItalicFont={KaiTi},BoldItalicFeatures={FakeBold=\FakeBoldValue}
    ]
%    \end{macrocode}
% 黑体字体。
% \changes{v0.1.3}{2022/01/03}{定义英文黑体字体族}
%    \begin{macrocode}
  \newCJKfontfamily\heiti{SimHei}[BoldFont={SimHei},BoldFeatures={FakeBold=\FakeBoldValue}]
  \newfontfamily\enheiti{SimHei}[BoldFont={SimHei},BoldFeatures={FakeBold=\FakeBoldValue}]
\fi
%    \end{macrocode}
% \subsubsection{标题样式}
% \changes{v1.0.2}{2022/03/09}{移除各级标题粗体样式}
%    \begin{macrocode}
\ctexset{
  section={
    name={,、},
    number={\chinese{section}},
    format={\raggedright\zihao{-4}},
    aftername={},
    beforeskip={0pt plus 0pt minus 0pt},
    afterskip={0pt plus 0pt minus 0pt}
  }
}
\ctexset{
  subsection={name={(,)},
    number={\chinese{subsection}},
    format={\raggedright\zihao{-4}},
    aftername={},
    titleformat={},
    beforeskip={8pt plus 0pt minus 0pt},
    afterskip={0pt plus 0pt minus 0pt}
  }
}
\ctexset{
  subsubsection={name={(,)},
    number={\arabic{subsubsection}},
    format={\raggedright\zihao{-4}},
    aftername={},
    titleformat={},
    beforeskip={8pt plus 0pt minus 0pt},
    afterskip={0pt plus 0pt minus 0pt},
    indent={2em}
  }
}
%    \end{macrocode}
% \subsubsection{封面和扉页}
% \begin{macro}{\ulthickness}
% 设置下划线粗细。
%    \begin{macrocode}
\newcommand\ulthickness{1pt}
%    \end{macrocode}
% \end{macro}
% \begin{macro}{\valueWithUL}
% 自定义固定长度下划线命令。
%    \begin{macrocode}
\newcommand\valueWithUL[2]{\CJKunderline[thickness=\ulthickness]{\makebox[#1]{#2}}}
%    \end{macrocode}
% \end{macro}
% 绘制封面和扉页。
%    \begin{macrocode}
\AtBeginDocument{
  \begin{titlepage}
  \vspace*{-30pt}
%    \end{macrocode}
% \paragraph{标题信息}
% \changes{v0.1.3}{2022/01/03}{修正标题年份数字字体}
%    \begin{macrocode}
  \begin{center}
  \zihao{-2}
  西安电子科技大学\CJKunderline[thickness=\ulthickness]{\school}学院
  \end{center}
  \vspace*{35pt}
  \begin{center}
  \heiti\zihao{2}
  本科生毕业论文(设计)开题报告
  \end{center}
  \vspace*{-23pt}
  \begin{center}
  \enheiti\heiti\zihao{-3}
  (\class{}届)
  \end{center}
%    \end{macrocode}
% \paragraph{个人信息}
%    \begin{macrocode}
  \vspace*{123pt}
  \begin{center}
  \zihao{-3}
%    \end{macrocode}
% \begin{macro}{\metaValueWidth}
% 存储最长的值的宽度。
%    \begin{macrocode}
  \newlength\metaValueWidth
%    \end{macrocode}
% \end{macro}
% \begin{macro}{\authorNameWidth,\majorWidth,\studentNoWidth,\advisorNameWidth}
% 存储首页若干值的宽度便于绘制下划线。
%    \begin{macrocode}
  \newlength\authorNameWidth
  \newlength\majorWidth
  \newlength\studentNoWidth
  \newlength\advisorNameWidth
  \settowidth{\authorNameWidth}{\authorName}
  \settowidth{\majorWidth}{\major}
  \settowidth{\studentNoWidth}{\studentNo}
  \settowidth{\advisorNameWidth}{\advisorName}
%    \end{macrocode}
% \end{macro}
% 计算最大值。
%    \begin{macrocode}
  \setlength{\metaValueWidth}{
  \maxof{\authorNameWidth}{
  \maxof{\majorWidth}{
  \maxof{\studentNoWidth}{\advisorNameWidth}}}}
%    \end{macrocode}
% 补充\valuex{2em}长度,使得下划线略长于文本。
%    \begin{macrocode}
  \addtolength{\metaValueWidth}{2em}
%    \end{macrocode}
% 增加每一项数据间的垂直距离。
%    \begin{macrocode}
  \renewcommand{\arraystretch}{1.5}
%    \end{macrocode}
% 排版开题报告元数据。
%    \begin{macrocode}
  \begin{tabular}{cc}
  学生姓名&\valueWithUL{\metaValueWidth}{\authorName}\\
  专\qquad{}业&\valueWithUL{\metaValueWidth}{\major}\\
  学\qquad{}号&\valueWithUL{\metaValueWidth}{\studentNo}\\
  指导教师&\valueWithUL{\metaValueWidth}{\advisorName}\\
  \end{tabular}
  \end{center}
  \vspace{80pt}
  \begin{center}
  \zihao{4}
  \submitDate
  \end{center}
  \vspace{50pt}
  \begin{center}
  \zihao{5}
  (本表一式三份,学生、指导教师、学院各一份)
  \end{center}
  \end{titlepage}
  \newpage
}
%    \end{macrocode}
% \subsubsection{绘制边框}
% \paragraph{定义边框样式}
% \changes{v0.1.1}{2022/01/03}{使用\tnx{newlength}替换\tnx{newcommand}来定义边框边距和线宽}
% \begin{macro}{\boxmargin,\boxrule}
% 定义边框边距和线宽。
% \changes{v1.0.1}{2022/03/09}{修正边框宽度}
%    \begin{macrocode}
\newlength\boxmargin
\setlength{\boxmargin}{5pt}
\newlength\boxrule
\setlength{\boxrule}{0.5pt}
%    \end{macrocode}
% \end{macro}
% \changes{v0.1.1}{2022/01/03}{适配最新版\pkgx{tcolorbox}宏包}
% \changes{v0.1.1}{2022/01/03}{增加\pkgx{tcolorbox}默认配置}
% 定义默认边框样式。
%    \begin{macrocode}
\tcbset{
  standard jigsaw,
  sharp corners=all,
  colframe=black,
  opacityback=0,
  boxsep=0pt,
  boxrule=\boxrule,
  top=\boxmargin,
  bottom=\boxmargin,
  left=\boxmargin,
  right=\boxmargin,
  beforeafter skip=0pt,
  before upper={\setlength{\parindent}{2em}\linespread{1}\fontsize{12}{20}\selectfont}
}
%    \end{macrocode}
% \begin{environment}{mybox1}
% 定义可分页且不撑满的边框,主要用于:
% \begin{enumerate}
% \item 论文名称及项目来源。
% \item 工作的主要阶段、进度和完成时间。
% \end{enumerate}
% \changes{v0.1.1}{2022/01/03}{移除\envx{mybox1}公共配置}
%    \begin{macrocode}
\newtcolorbox{mybox1}{
  breakable,
  height fixed for=first and middle
}
%    \end{macrocode}
% \end{environment}
% \begin{environment}{mybox2}
% 定义可分页且撑满的边框,主要用于:
% \begin{enumerate}
% \item 研究目的和意义。
% \item 国内外研究现状和发展趋势。
% \item 主要研究内容、要解决的问题及本文的初步方案。
% \item 已进行的前期准备工作。
% \end{enumerate}
% \changes{v0.1.1}{2022/01/03}{移除\envx{mybox2}公共配置}
%    \begin{macrocode}
\newtcolorbox{mybox2}{
  breakable,
  height fixed for=all,
  height fill=maximum
}
%    \end{macrocode}
% \end{environment}
% \begin{environment}{mybox3}
% 定义高度仅有半页的边框,主要用于:
% \begin{enumerate}
% \item 指导教师意见。
% \item 学院审核意见。
% \end{enumerate}
% \changes{v0.1.1}{2022/01/03}{移除\envx{mybox3}公共配置}
%    \begin{macrocode}
\newtcolorbox{mybox3}{
  height=.5\textheight+.5\boxrule,
  space to upper,
  lower separated=false,
  halign lower=flush right
}
%    \end{macrocode}
% \end{environment}
% \paragraph{定义边框环境}
% \begin{environment}{mcly}
% 论文名称及项目来源。
%    \begin{macrocode}
\newenvironment{mcly}
  {\begin{mybox1}}
  {\end{mybox1}
  \vspace*{-\boxrule}}
%    \end{macrocode}
% \end{environment}
% \begin{environment}{yjmdyy}
% 研究目的和意义。
% \changes{v0.1.1}{2022/01/03}{调整\envx{yjmdyy}上边距}
%    \begin{macrocode}
\newenvironment{yjmdyy}
  {\begin{mybox2}}
  {\end{mybox2}}
%    \end{macrocode}
% \end{environment}
% \begin{environment}{yjxz}
% 国内外研究现状和发展趋势。
%    \begin{macrocode}
\newenvironment{yjxz}
  {\begin{mybox2}}
  {\vspace*{8pt}
  \ckwx
  \bibliography{xduugtp}
  \end{mybox2}}
%    \end{macrocode}
% \end{environment}
% \begin{environment}{yjnr}
% 主要研究内容、要解决的问题及本文的初步方案。
%    \begin{macrocode}
\newenvironment{yjnr}
  {\begin{mybox2}}
  {\end{mybox2}}
%    \end{macrocode}
% \end{environment}
% \begin{environment}{gzjd}
% 工作的主要阶段、进度和完成时间。
% \changes{v0.1.1}{2022/01/03}{调整\envx{gzjd}下边距}
%    \begin{macrocode}
\newenvironment{gzjd}
  {\begin{mybox1}}
  {\end{mybox1}
  \vspace*{-\boxrule}}
%    \end{macrocode}
% \end{environment}
% \begin{environment}{zbgz}
% 已进行的前期准备工作。
%    \begin{macrocode}
\newenvironment{zbgz}
  {\begin{mybox2}}
  {\end{mybox2}}
%    \end{macrocode}
% \end{environment}
% \begin{environment}{zdjsyj}
% 指导教师意见。
% \changes{v0.3.0}{2022/01/10}{在\envx{zdjsyj}处插入手写日期图像}
%    \begin{macrocode}
\newenvironment{zdjsyj}
  {\begin{mybox3}}
  {\tcblower\linespread{1}\fontsize{12}{30}\selectfont
  签名\quad\makebox[5em][l]{\zdjsqm}\\
  \zdjsqmrq
  \end{mybox3}}
%    \end{macrocode}
% \end{environment}
% \begin{environment}{xyshyj}
% 学院审核意见。
% \changes{v0.3.0}{2022/01/10}{在\envx{xyshyj}处插入手写日期图像}
%    \begin{macrocode}
\newenvironment{xyshyj}
  {\vspace*{-\boxrule}
  \begin{mybox3}}
  {\tcblower\linespread{1}\fontsize{12}{30}\selectfont
  签名\quad\makebox[5em][l]{\xyshqm}\\
  \xyshqmrq
  \end{mybox3}}
%    \end{macrocode}
% \end{environment}
%    \begin{macrocode}
%</class>
%    \end{macrocode}
% \subsection{xduugtp.tex}
%    \begin{macrocode}
%<*tex>
%    \end{macrocode}
%    \begin{macrocode}
\documentclass{xduugtp}
%% 学院
\newcommand\school{电子工程}
%% 界
\newcommand\class{2022}
%% 姓名
\newcommand\authorName{张三}
%% 专业
\newcommand\major{电子科学与技术}
%% 学号
\newcommand\studentNo{1101110071}
%% 指导教师
\newcommand\advisorName{李四}
%% 开题日期
\newcommand\submitDate{2022年1月2日}
%% 签名图像
%% 指导教师签名及日期
%% \renewcommand\zdjsqm{\sign{lisi}}
%% \renewcommand\zdjsqmrq{\sign{lisidate}}
%% \renewcommand\zdjsqmrq{2022年1月10日}
%% 学院签名及日期
%% \renewcommand\xyshqm{\sign{lisi}}
%% \renewcommand\xyshqmrq{\sign{lisidate}}
%% \renewcommand\xyshqmrq{2022年1月10日}
%    \end{macrocode}
%    \begin{macrocode}
\begin{document}
\begin{mcly}
\section{论文名称及项目来源}
%% 在这里撰写论文名称及项目来源
\end{mcly}
\begin{yjmdyy}
\section{研究目的和意义}
%% 在这里撰写研究目的和意义
\end{yjmdyy}
\begin{yjxz}
\section{国内外研究现状和发展趋势}
%% 在这里撰写国内外研究现状和发展趋势
\end{yjxz}
\begin{yjnr}
\section{主要研究内容、要解决的问题及本文的初步方案}
%% 在这里撰写主要研究内容、要解决的问题及本文的初步方案
\end{yjnr}
\begin{gzjd}
\section{工作的主要阶段、进度和完成时间}
%% 在这里撰写工作的主要阶段、进度和完成时间
\end{gzjd}
\begin{zbgz}
\section{已进行的前期准备工作}
%% 在这里撰写已进行的前期准备工作
\end{zbgz}
\begin{zdjsyj}
\section{指导教师意见}
%% 在这里撰写指导教师意见
\end{zdjsyj}
\begin{xyshyj}
\section{学院审核意见}
%% 在这里撰写学院审核意见
\end{xyshyj}
\end{document}
%    \end{macrocode}
%    \begin{macrocode}
%</tex>
%    \end{macrocode}
% \subsection{xduugtp.bib}
%    \begin{macrocode}
%<*bib>
%    \end{macrocode}
%    \begin{macrocode}
@article{ChangHTD19,
author = {Yuan{-}Hao Chang and Jingtong Hu and Mehdi Baradaran Tahoori and Ronald F. DeMara},
title = {Guest Editorial: {IEEE} Transactions on Computers Special Section on Emerging Non-Volatile Memory Technologies: From Devices to Architectures and Systems},
journal = {{IEEE} Trans. Computers},
volume = {68},
number = {8},
pages = {1111--1113},
year = {2019}
}
@article{WangZSS21,
author = {王鹏 and 张修社 and 索龙 and 史可懿},
title = {基于随机时变图的时间确定性网络路由算法},
journal = {通信学报},
volume = {42},
number = {9},
pages = {21--30},
year = {2021}
}
@article{GongL21,
author = {Ming{-}Yan Gong and Bin Lyu},
title = {Alternating Maximization and the {EM} Algorithm in Maximum-Likelihood Direction Finding},
journal = {{IEEE} Trans. Veh. Technol.},
year = {2021},
doi = {10.1109/TVT.2021.3106794}
}
@inproceedings{TsaiCLQLB13,
author = {Wei{-}Tek Tsai and Charles J. Colbourn and Jie Luo and Guanqiu Qi and Qingyang Li and Xiaoying Bai},
title = {Test algebra for combinatorial testing},
booktitle = {{AST}},
pages = {19--25},
publisher = {{IEEE} Computer Society},
year = {2013}
}
@online{Collinson21,
author = {Stephen Collinson},
title = {Biden's refusal of executive privilege claim ignites new firestorm with {Trump}},
url = {https://edition.cnn.com/2021/10/26/politics/donald-trump-joe-biden-executive-privilege-january-6/index.html},
urldate = {2021-10-26}
}
%    \end{macrocode}
%    \begin{macrocode}
%</bib>
%    \end{macrocode}
% \subsection{latexmkrc}
%    \begin{macrocode}
%<*latexmkrc>
%    \end{macrocode}
%    \begin{macrocode}
$pdf_mode = 5;
$xelatex = 'xelatex -synctex=1';
$clean_ext = 'aux bbl blg fdb_latexmk fls lof log lot out synctex(busy) synctex.gz toc xdv';
@default_files = ('xduugtp');
%    \end{macrocode}
%    \begin{macrocode}
%</latexmkrc>
%    \end{macrocode}
% \Finale
\endinput
